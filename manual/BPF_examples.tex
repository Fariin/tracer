\section{\IFRU{Примеры}{Examples}}

\subsection{\IFRU{Простое использование}{Simple usage}}

\begin{lstlisting}
tracer.exe -l:bzip2.exe bpf=.*!fprintf,args:3
\end{lstlisting}

\begin{lstlisting}
TID=5128|(0) cygwin1.dll!fprintf (0x61103150, "%s: I won't write compressed data to a terminal.\n", "bzip2") (called from 0x401e03 (bzip2.exe!BASE+0x1e03))
TID=5128|(0) cygwin1.dll!fprintf -> 0x34
TID=5128|(0) cygwin1.dll!fprintf (0x61103150, "%s: For help, type: `%s --help'.\n", "bzip2") (called from 0x401c66 (bzip2.exe!BASE+0x1c66))
TID=5128|(0) cygwin1.dll!fprintf -> 0x27
\end{lstlisting}

\subsection{\IFRU{Перехват некоторых Windows-функций для работы с реестром}{Intercept some Windows registry access functions}}

\begin{lstlisting}
tracer.exe -l:someprocess.exe bpf=advapi32.dll!RegOpenKeyExA,args:5 bpf=advapi32.dll!RegQueryValueExA,args:6 bpf=advapi32.dll!RegSetValueExA,args:6
\end{lstlisting}

.. \IFRU{или измените суффиксы функция на W и добавьте опцию UNICODE}{or change function suffixes to W and add UNICODE option}:

\begin{lstlisting}
tracer64.exe -l:far.exe bpf=advapi32.dll!RegOpenKeyExW,args:5,unicode bpf=advapi32.dll!RegQueryValueExW,args:6,unicode bpf=advapi32.dll!RegSetValueExW,args:6,unicode
\end{lstlisting}

\subsection{\IFRU{Подавить шумный сигнал}{Suppress noisy beeping}}

\begin{lstlisting}
tracer.exe -l:beeper.exe bpf=kernel32.dll!Beep,args:2,skip_stdcall,rt:1
\end{lstlisting}

\subsection{\IFRU{Подавить диалоговое окно с сообщением}{Suppress Message Box}}

... \IFRU{и сделать так что вызываемая функция будет считать что пользователь каждый раз нажимает OK (константа IDOK равняется 1)}{by making it appear to a caller that the user presses OK every time (IDOK constant is 1)}:

\begin{lstlisting}
tracer.exe -l:filename.exe bpf=user32.dll!MessageBoxA,args:4,skip_stdcall,rt:1
\end{lstlisting}

... \IFRU{или CANCEL (константа IDCANCEL равняется 2)}{or CANCEL (IDCANCEL constant is 2)}:

\begin{lstlisting}
tracer.exe -l:filename.exe bpf=user32.dll!MessageBoxA,args:4,skip_stdcall,rt:2
\end{lstlisting}

\subsection{\IFRU{Перехват вызовов rand()}{Intercepting rand() call}}

\IFRU{Бывает весело перехватывать вызовы функции rand() в различных играх. Например, пасьянс Solitaire в Windows использует его для того чтобы сгенерировать случайный расклад. Мы можем установить возвращаемое значение rand() в ноль, и тогда Solitaire будет раздавать один и тот же расклад, всегда:}{Another fun is intercepting rand() function in various games. For example, Windows Solitaire card game use it to generate random deal. We can fix rand() return at zero, and Solitaire will do the same deal each time, forever:}

\IFRU{В}{In} Windows XP x86/x64:

\begin{lstlisting}
tracer.exe/tracer64.exe -l:c:\windows\system32\sol.exe bpf=.*!rand,rt:0
\end{lstlisting}

\IFRU{В}{In} Windows 7 x64:

\begin{lstlisting}
tracer64.exe -l:[full path to]\Solitaire.exe bpf=.*!rand,rt:0
\end{lstlisting}

\subsection{FreeCell}

\IFRU{Когда вы запускаете FreeCell в Windows (XP SP3) и нажимаете F2 (Новая игра), вы видите сообщение "Do you want to resign this game?" Мы можем подавить звуковой сигнал и сделать так что FreeCall будет думать что пользователь всегда нажимает YES:}{When you run Windows (XP SP3) FreeCell and press F2 (New game), you will get a message box "Do you want to resign this game?" We can suppress all that beeping and also make illusion to FreeCell user always press YES:}

\IFRU{Константа IDYES - 6. FreeCell использует функцию MessageBoxW - суффикс W означает уникодную версию функции MessageBox.}{IDYES constant is 6. FreeCell use MessageBoxW - W mean unicode version of MessageBox.}

\IFRU{В}{In} Windows XP SP3 x86:

\begin{lstlisting}
tracer.exe -l:c:\windows\system32\freecell.exe bpf=user32.dll!messagebeep,args:1,skip_stdcall bpf=user32.dll!messageboxw,args:4,unicode,skip_stdcall,rt:6
\end{lstlisting}

\begin{lstlisting}
TID=444|(0) USER32.dll!MessageBeep (0x20) (called from 0x1001f52 (freecell.exe!BASE+0x1f52))
We skip execution of this function
TID=444|(1) USER32.dll!MessageBoxW (0x80122, L"Do you want to resign this game?", L"FreeCell", 0x24) (called from 0x1001f5f (freecell.exe!BASE+0x1f5f))
We skip execution of this function
TID=444|We modify return value (EAX) of this function to 6
\end{lstlisting}

\IFRU{В}{In} Windows XP SP2 x64 Russian:

\begin{lstlisting}
tracer64.exe -l:c:\windows\system32\freecell.exe bpf=user32.dll!messagebeep,args:1,skip bpf=user32.dll!messageboxw,args:4,unicode,skip,rt:6
\end{lstlisting}

\begin{lstlisting}
TID=2028|(0) USER32.dll!MessageBeep (0x20) (called from 0x1000023f9 (freecell.exe!BASE+0x23f9))
We skip execution of this function
TID=2028|(1) USER32.dll!MessageBoxW (0x1f00f0, 0xbf80, 0xbf20, 0x24) (called from 0x100002416 (freecell.exe!BASE+0x2416))
We skip execution of this function
TID=2028|We modify return value (RAX) of this function to 6
\end{lstlisting}

\subsection{\IFRU{Проверка ивентов и запись в лог в Oracle RDBMS}{Oracle RDBMS Events checking and log writes}}

\IFRU{В}{In} Oracle 10.2.0.1 win64:

\begin{lstlisting}
tracer64.exe -a:oracle.exe bpf=oracle.exe!ksdpec,args:1 bpf=oracle.exe!ss_wrtf,args:3
\end{lstlisting}

( \IFRU{Смотрите также}{See also}: \url{http://blog.yurichev.com/node/14} )

\begin{lstlisting}
TID=3032|(0) oracle.exe!ksdpec (0x2743) (called from 0x9580a9 (oracle.exe!opiodr+0x105))
TID=3032|(0) oracle.exe!ksdpec -> 0xff
TID=3032|(1) oracle.exe!ss_wrtf (0x4a0, "*** 2009-12-04 06:19:01.005\n", 0x1b) (called from 0x45318d (oracle.exe!sdpri+0x22d))
TID=3032|(1) oracle.exe!ss_wrtf -> 1
TID=3032|(1) oracle.exe!ss_wrtf (0x4a0, "OPI CALL: type=107 argc= 3 cursor=  0 name=SES OPS (80)\n", 0x37) (called from 0x45318d (oracle.exe!sdpri+0x22d))
TID=3032|(1) oracle.exe!ss_wrtf -> 1
TID=3032|(0) oracle.exe!ksdpec (0x2743) (called from 0x9580a9 (oracle.exe!opiodr+0x105))
TID=3032|(0) oracle.exe!ksdpec -> 0xff
TID=3032|(1) oracle.exe!ss_wrtf (0x4a0, "OPI CALL: type=59 argc= 4 cursor=  0 name=VERSION2\n", 0x32) (called from 0x45318d (oracle.exe!sdpri+0x22d))
TID=3032|(1) oracle.exe!ss_wrtf -> 1
TID=3032|(0) oracle.exe!ksdpec (0x273e) (called from 0x4a00cc (oracle.exe!kslwte_tm+0x7a8))
TID=3032|(0) oracle.exe!ksdpec -> 0
TID=3032|(0) oracle.exe!ksdpec (0x273e) (called from 0x4a00cc (oracle.exe!kslwte_tm+0x7a8))
TID=3032|(0) oracle.exe!ksdpec -> 0
TID=3032|(0) oracle.exe!ksdpec (0x2743) (called from 0x9580a9 (oracle.exe!opiodr+0x105))
TID=3032|(0) oracle.exe!ksdpec -> 0xff
TID=3032|(1) oracle.exe!ss_wrtf (0x4a0, "OPI CALL: type=104 argc=12 cursor=  0 name=Transaction Commit/Rollback\n", 0x46) (called from 0x45318d (oracle.exe!sdpri+0x22d))
TID=3032|(1) oracle.exe!ss_wrtf -> 1
\end{lstlisting}

\subsection{\IFRU{Слежение за выделением памяти в}{Trace memory allocations in} Oracle 11.1.0.6.0 win32/win64}

\begin{lstlisting}
tracer.exe/tracer64.exe -a:oracle.exe bpf=.*!kghalf,args:6 bpf=.*!kghfrf,args:4
\end{lstlisting}

\begin{lstlisting}
TID=1600|(0) oracle.exe!kghalf (0x6d35af0, 0xb507ef8, 0x1000, 0, 0, "kzsrcrdi") (called from 0x1c7aa83 (oracle.exe!kzctxhugi+0x71))
TID=1600|(0) oracle.exe!kghalf -> 0xfa3ea58

TID=1600|(0) oracle.exe!kghalf (0x6d35af0, 0xb507ef8, 0x58, 1, 0x6d35530, "UPI heap") (called from 0x1e7f8b7 (oracle.exe!__PGOSF266_kwqmahal+0x5b))
TID=1600|(0) oracle.exe!kghalf -> 0xfa4d0d8

TID=1188|(0) oracle.exe!kghalf (0xda39540, 0xda39240, 0x88, 0, "ksirmdt array", 0xda39240) (called from 0x6afb5b (oracle.exe!ksz_nfy_ipga+0xf1))
TID=1188|(0) oracle.exe!kghalf -> 0x105d0b10

TID=1188|(0) oracle.exe!kghalf (0xda39540, 0xda39240, 0x48, 1, 0x1204e400, "local") (called from 0x3684a64 (oracle.exe!kjztcxini+0x58))
TID=1188|(0) oracle.exe!kghalf -> 0x105d0ab0
\end{lstlisting}

\subsection{\IFRU{Слежение за разбором SQL-выражений в}{SQL statements parsing in} Oracle RDBMS}

\IFRU{В}{In} Oracle 11.1.0.6.0 win32/win64:

\begin{lstlisting}
tracer.exe/tracer64.exe -a:oracle.exe bpf=oracle.exe!_?rpisplu,args:8 bpf=oracle.exe!_?kprbprs,args:7 bpf=oracle.exe!_?opiprs,args:6 bpf=oraclient11.dll!OCIStmtPrepare,args:6</i></p>
\end{lstlisting}

\IFRU{Замечание: регулярное выражение \IT{\_?function} покрывает оба имени: \TT{function} и \TT{\_function}.}{Note: regular expression \TT{\_?function} cover both \TT{function} and \TT{\_function}.}

\begin{lstlisting}
TID=1140|(2) oracle.exe!opiprs (0x13f029d0, "select 1 from obj$ where name='DBA_QUEUE_SCHEDULES'", 0x34, 0x10ae7f50, 0x840082, 0xd9f7a10) (called from 0x6ba3bf (oracle.exe!__PGOSF423_kksParseChildCursor+0x2dd))
TID=1140|(2) oracle.exe!opiprs -> 0
TID=1140|(2) oracle.exe!opiprs (0x13f029d0, "select 1 from sys.aq$_subscriber_table where rownum < 2 and subscriber_id <> 0 and table_objno <> 0", 0x64, 0x10ad5de8, 0, 0x13f007e0) (called from 0x6ba3bf (oracle.exe!__PGOSF423_kksParseChildCursor+0x2dd))
TID=1140|(2) oracle.exe!opiprs -> 0
TID=1140|(0) oracle.exe!rpisplu (3, 0, 0, 0, 0, 0x14430ac0, 0, 0) (called from 0x250b33c (oracle.exe!kqdGetCursor+0x106))
TID=1140|(0) oracle.exe!rpisplu -> 0
TID=1288|(2) oracle.exe!opiprs (0x17df8130, "select * from v$version", 0x18, 0x10adee60, 0, 0) (called from 0x6ba3bf (oracle.exe!__PGOSF423_kksParseChildCursor+0x2dd))
TID=1288|(1) oracle.exe!kprbprs (0xa82bc50, 0, "select timestamp, flags from fixed_obj$ where obj#=:1", 0x35, 0xffffe3e0, 0x2040800, 1) (called from 0x2ba1b1f (oracle.exe!kqldtstr+0x151))
TID=1288|(1) oracle.exe!kprbprs -> 0
TID=1288|(0) oracle.exe!rpisplu (0x1f, 0, 0, 0, 0, 0x2bb5e04, "select  BANNER from GV$VERSION where inst_id = USERENV('Instance')", 0xffffc085) (called from 0x2bbcabf (oracle.exe!kqldFixedTableLoadCols+0x157))
TID=1288|(1) oracle.exe!kprbprs (0x1090c108, 0, "select timestamp, flags from fixed_obj$ where obj#=:1", 0x35, 0xffffe3e0, 0x2040800, 1) (called from 0x2ba1b1f (oracle.exe!kqldtstr+0x151))
TID=1288|(1) oracle.exe!kprbprs -> 0
TID=1288|(1) oracle.exe!kprbprs (0x10908060, 0, "select timestamp, flags from fixed_obj$ where obj#=:1", 0x35, 0xffffe3e0, 0x2040800, 1) (called from 0x2ba1b1f (oracle.exe!kqldtstr+0x151))
TID=1288|(1) oracle.exe!kprbprs -> 0
TID=1288|(2) oracle.exe!opiprs -> 0
TID=1288|(0) oracle.exe!rpisplu -> 0
TID=1288|(0) oracle.exe!rpisplu (0x16, 0, 0, 0, 0, 0x10b3ce50, 0, 0) (called from 0x250b33c (oracle.exe!kqdGetCursor+0x106))
TID=1288|(0) oracle.exe!rpisplu -> 0
\end{lstlisting}

\subsection{\IFRU{Игнорирование неподписанных драйверов}{Ignore unsigned drivers}}

\begin{lstlisting}
tracer.exe -l:target.exe bpf=Wintrust.dll!WinVerifyTrust,rt:0
\end{lstlisting}

\subsection{\IFRU{Вывод памяти по аргументам функций}{Dump function arguments}}

\begin{lstlisting}
tracer.exe -l:rar.exe "-c:a archive.rar *.exe" bpf=kernel32.dll!writefile,args:5,dump_args:0x10
\end{lstlisting}

\IFRU{RAR записывает свою сигнатуру в начало файла archive.rar:}{RAR writting its signature to the beginning of archive.rar file:}

\begin{lstlisting}
TID=7000|(0) KERNEL32.dll!WriteFile (0x118, 0x152410, 7, 0x150fc0, 0) (called from 0x403721 (rar.exe!__GetExceptDLLinfo+0x26c8))
Dump of buffer at argument 2 (starting at 1)
00152410: 52 61 72 21 1A 07 00 00-50 30 15 00 5D 83 40 00 "Rar!....P0..].@."
Dump of buffer at argument 4 (starting at 1)
00150FC0: 00 00 00 00 21 7B 40 00-10 24 15 00 18 24 15 00 "....!{@..$...$.."
TID=7000|(0) KERNEL32.dll!WriteFile -> 1
\end{lstlisting}

\subsection{\IFRU{Вывод памяти по аргументам функций и слежение за её изменением}{Dump function arguments and track difference occured in buffers}}

\begin{lstlisting}
tracer.exe -l:rar.exe "-c:x archive.rar" bpf=kernel32.dll!readfile,args:4,dump_args:0x10
\end{lstlisting}

\IFRU{Архиватор RAR открывает файл archive.rar и первым делом читает сигнатуру:}{RAR archiver open archive.rar and read signature for the first:}

\begin{lstlisting}
TID=6148|(0) KERNEL32.dll!ReadFile (0x120, 0x17b3f8, 7, 0x174c50) (called from 0x403966 (rar.exe!__GetExceptDLLinfo+0x290d))
Dump of buffer at argument 2 (starting at 1)
0017B3F8: 00 00 00 00 00 00 00 00-00 00 00 00 48 00 00 00 "............H..."
Dump of buffer at argument 4 (starting at 1)
00174C50: 07 00 00 00 78 4C 17 00-7A 38 40 00 8C 6D 17 00 "....xL..z8@..m.."
TID=6148|(0) KERNEL32.dll!ReadFile -> 1
Dump difference of buffer at argument 2 (starting at 1)
00000000: 52 61 72 21 1A 07      -                        "Rar!..          "
\end{lstlisting}

